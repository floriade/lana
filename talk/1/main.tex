\documentclass[black,white]{beamer}

\usepackage{beamerthemesplit}
\usepackage{epstopdf}
\usepackage{graphicx}
\usepackage{subfigure}
\usepackage{hyperref}
\usepackage[utf8]{inputenc}
\usepackage[english]{babel}
\usepackage{listings}

\usefonttheme{professionalfonts}
\usecolortheme{dove}
\useoutertheme{infolines}
\useinnertheme{rectangles}

\setlength{\parindent}{0pt}
\newcommand*\sfb[1]{\textbf{#1}}
\definecolor{red2}{rgb}{.7,0,.39}
\usebackgroundtemplate{
	\includegraphics[width=\paperwidth,height=\paperheight]{img/bg.png}
}

\begin{document}

\title[LANA for Embedded Devices]{\newline \newline \normalsize{Master Thesis} \newline
				  \huge{\textcolor{red2}{Lightweight Autonomic Network Architecture for Embedded Devices}}}
\author[Daniel Borkmann] {
	\vspace*{-20pt}
	\newline
	Daniel Borkmann	\texttt{<dborkma@tik.ee.ethz.ch>}\\
	\footnotesize{intermediate presentation}
}
\institute[ETH Zurich] {
	Communication Systems Group\\
	ETH Zurich\\\bigskip
	Faculty of Computer Science, Mathematics and Natural Sciences\\
	Leipzig University of Applied Sciences
}
\date[\today]{}

\frame {
	\titlepage
}

\frame {
	\frametitle{\textcolor{red2}{Communication Systems Group, \newline ETH Zurich}}
	\bigskip
	\begin{itemize}
		\item Communication Systems Group conducts \textbf{research on modeling, design, and implementation of communication systems} [1]:\medskip
		\begin{itemize}
			\item Wireless mobile networks and social networks\medskip
			\item Network measurements and security\medskip
			\item Future Internet architecture and protocols\medskip
		\end{itemize}
		\item \textbf{Thesis advisors:} Ariane Keller, Dr. Wolfgang Mühlbauer\medskip
		\item \textbf{Professor:} Prof. Dr. Bernhard Plattner
	\end{itemize}
}

\frame {
	\frametitle{\textcolor{red2}{Table of Contents}}
	\tableofcontents
}

\section{Introduction}
\frame {
	\frametitle{\textcolor{red2}{Introduction}}
	\bigskip
	\begin{itemize}
		\item Thesis in \textbf{context of the EPiCS} research project [2]\medskip
		\begin{itemize}
			\item Concepts and foundations for self-aware and self-expressive systems\medskip
			\item Hardware/software platform technologies for autonomic compute nodes\medskip
			\item Self-aware network architectures\medskip
		\end{itemize}
		\item For communication purposes, the \textbf{Autonomic Network Architecture (ANA)} [3] will be used
	\end{itemize}
}

\section{Basics of ANA}
\frame {
	\frametitle{\textcolor{red2}{Basic idea of ANA}}
	\bigskip
	\bigskip
	\begin{itemize}
		\item \textbf{No "one-size-fits-all" approach} as in classical network architectures\medskip
		\item Network stack is composed by building block processing elements, called \textbf{Functional Blocks (FB)}\medskip
		\item Idea of flexible UNIX Sockets for communication and packet forwarding between processing elements\medskip
		\item Results in a graph of processing elements where the edges represent communication connections\medskip
		\item Packets traverse specific paths of this graph
	\end{itemize}
}

\frame {
	\frametitle{\textcolor{red2}{Advantages of this idea}}
	\bigskip
	\begin{itemize}
		\item \textbf{Smaller size of the network stack}\medskip
		\item \textbf{Network stack reconfiguration during runtime}\medskip
		\item Processing elements need no a-priori knowledge about their "neighboring" elements\medskip
		\item Easier to adapt new technologies and protocols\medskip
		\item Application developers do not need to care about underlying addressing schemes, i.e. IPv4 versus IPv6\medskip
		\item \textbf{Not only suited for Internet!}
	\end{itemize}
}

\section{Motivation}
\frame {
	\frametitle{\textcolor{red2}{Motivation}}
	\bigskip
	\bigskip
	\begin{itemize}
		\item \textbf{Autonomic Network Architecture important for \newline Embedded Systems [4]}\medskip
		\item Limited resources require network stack adaptions to \newline different networking situations\medskip
		\item Implementation of ANA rather resource-intensive, lightweight \newline redesign is needed\medskip
		\item \textit{Short example:}\medskip
		\begin{itemize}
			\item Currently, packets are being copied between Functional Blocks
			\item Can be avoided to save CPU processing resources
		\end{itemize}
	\end{itemize}
}

\section{Aims of this thesis}
\frame {
	\frametitle{\textcolor{red2}{Aims of this thesis}}
	\bigskip
	\begin{itemize}
		\item \textbf{Lightweight redesign and implementation of ANA (LANA)}\medskip
		\item \textit{Requirements:}\medskip
		\begin{itemize}
			\item Needs to outperform the original architecture regarding performance\medskip
			\item Needs to run on embedded Linux devices\medskip
			\item Functional Block adding/removal/swapping during runtime
		\end{itemize}
	\end{itemize}
}

\section{Fundamental changes in LANA}
\frame {
	\frametitle{\textcolor{red2}{Fundamental changes in LANA}}
	% core machinery
}

\frame {
	\frametitle{\textcolor{red2}{Fundamental changes in LANA}}
	% architectural changes
}

\section{Architecture overview}
\frame {
	\frametitle{\textcolor{red2}{Architecture overview}}
}

\section{Performance Evaluation and Testing Tools}
\frame {
	\frametitle{\textcolor{red2}{Performance Evaluation and \newline Testing Tools}}
}

\section{Open Questions}
\frame {
	\frametitle{\textcolor{red2}{Open Questions}}
}

\section{Project Plan}
\frame {
	\frametitle{\textcolor{red2}{Project Plan}}
}

\section{Questions}
\frame{
	\frametitle{\textcolor{red2}{Questions}}
	\bigskip
	\bigskip
	\begin{center}
		\Large{Thanks for your attention! Questions?}
	\end{center}
}

\section{Resources}
\frame{
	\frametitle{\textcolor{red2}{Resources}}
	\bigskip
	\bigskip
	\medskip
	\begin{description}
		\item[ 1 ]\textbf{CSG, ETH Zurich}, \textcolor{blue}{\url{http://www.csg.ethz.ch/}} (3/2011)\medskip
		\item[ 2 ]\textbf{EPiCS project}, \textcolor{blue}{\url{http://www.epics-project.eu/project.html}} (3/2011)\medskip
		\item[ 3 ]\textbf{ANA project}, \textcolor{blue}{\url{http://www.ana-project.org/}} (3/2011)\medskip
		\item[ 4 ]\textbf{Reconfigurable nodes for future networks}, A. Keller, B. Plattner,
			E. Lübbers, M. Platzner, and C. Plessl in Proc. IEEE Globecom Workshop
			on Network of the Future (FutureNet), pages 372–376. IEEE, Dec. 2010\medskip
		\item[ 5 ]\textbf{netsniff-ng}, \textcolor{blue}{\url{http://www.netsniff-ng.org/}} (3/2011)
	\end{description}
}

\end{document}

